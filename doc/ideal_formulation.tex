\documentclass{article}
\usepackage{amsmath, amsfonts}
\title{An ideal non-extended formulation for leaky ReLU}
\begin{document}
\maketitle
\section{Objective}
For a leaky ReLU unit
\begin{subequations}
\begin{align}
	y = \max(c(w^Tx+b), w^Tx+b)\\
	L \le x \le U
\end{align}
\end{subequations}
where $0\le c < 1$, we will find an ideal non-extended formulation. Please refer to \cite{Anderson2020} for an explanation on the ideal formulation. In \cite{Anderson2020} they derived the ideal formulation for a ReLU neuron $y = \max(0, w^Tx+b)$, here we extend their approach to leaky ReLU neuron.

\section{Big-M formulation}
If we denote $m^+ = \sum_i \max(w_iU_i, w_iL_i) + b_i, m^- = \sum_i \min(w_iU_i, w_iL_i) + b_i$, then big-M formulation is
\begin{subequations}
\begin{align}
	y \ge c(w^Tx+b)\\
	y \ge w^Tx+b\\
    y \le c(w^Tx+b) + (1-c)m^+\beta\\
    y \le w^Tx+b - (1-c)m^-(1-\beta)\\
    L \le x \le U\\
\end{align}
\end{subequations}
This formulation is not ideal, as explained in Fig 1 of \cite{Anderson2020}.

\section{Ideal extended formulation}
We consider an ideal extended formulation with new slack continuous variables
\begin{subequations}
\begin{align}
	x = x^0 + x^1\\
	y = y^0 + y^1\\
	y^0 = c(w^Tx^0 + b(1-\beta)) \le 0\\
	y^1 = w^Tx^1 + b\beta\ge 0\\
	L(1-\beta) \le x^0 \le U(1-\beta)\\
	L\beta \le x^1 \le U\beta
\end{align}
\end{subequations}
This ideal formulation is extended as it introduces new slack variables $x^0, x^1, y^0, y^1$.

\section{Ideal non-extended formulation}
To derive the ideal non-extended formulation, we want to remove the slack variables $x^0, x^1, y^0, y^1$ in the ideal extended formulation. We first write $x^1, y^0, y^1$ as function of $x, \beta, x^0$
\begin{subequations}
\begin{align}
	x^1 = x - x^0\\
	y^0 = c(w^Tx^0 + b(1-\beta))\\
	y^1 = w^Tx^1 + b\beta = w^Tx - w^Tx^0 + b\beta\\
	y = y^0 + y^1 = w^Tx - (1-c)w^Tx^0 + cb + b(1-c)\beta
\end{align}
\end{subequations}
Hence we can express $(1-c)w^1x^0_1$ as
\begin{align}
	(1-c)w_1x^0_1 = w^Tx - (1-c)\sum_{i>1}w_ix^0_i + cb  + b(1-c)\beta - y\label{eq:1_c_w1_x01}
\end{align}
We define 
\begin{align}
	\bar{L}_i = \begin{cases}L_i\text{ if } w_i > 0\\U_i\text{ if } w_i < 0\end{cases}
	\bar{U}_i = \begin{cases}U_i\text{ if } w_i > 0\\L_i\text{ if } w_i < 0\end{cases}
\end{align}
We have
\begin{subequations}
\begin{align}
	(1-c)w_1\bar{L}_1(1-\beta) \le (1-c)w_1x^0_1\le (1-c)w_1\bar{U}_1(1-\beta)\\
	(1-c)w_1(x_1-\bar{U}_1\beta)\le (1-c)w_1x^0_1\le(1-c)w_1(x_1-\bar{L}_1\beta)
\end{align}
\end{subequations}
replacing $(1-c)w_1x^0_1$ with the right-hand side of \eqref{eq:1_c_w1_x01} we get
\begin{subequations}
	\begin{align}
		y \le w^Tx + cb + b(1-c)\beta - (1-c)w_1\bar{L}_1(1-\beta)-(1-c)\sum_{i>1}w_ix^0_i\\
		y \le cw_1x_1 + \sum_{i>1}w_ix_i + cb + b(1-c)\beta + (1-c)w_1\bar{U}_1\beta - (1-c)\sum_{i>1}w_ix^0_i\\
		y \ge w^Tx + cb + b(1-c)\beta - (1-c)w_1\bar{U}_1(1-\beta)-(1-c)\sum_{i>1}w_ix^0_i\\
		y \ge cw_1x_1 + \sum_{i>1}w_ix_i + cb + b(1-c)\beta + (1-c)w_1\bar{L}_1\beta - (1-c)\sum_{i>1}w_ix^0_i
	\end{align}
\end{subequations}
Now we proceed to get rid of $x^0_2$, we have
\begin{align}
	\begin{split}
	\begin{cases}
		w^Tx + cb + b(1-c)\beta - (1-c)w_1\bar{U}_1(1-\beta) - (1-c)\sum_{i>2}w_ix^0_i - y\\
		cw_1x_1 + \sum_{i>1}w_ix_i + cb + b(1-c)\beta + (1-c)\bar{L}_1\beta - (1-c)\sum_{i>2}w_ix^0_i-y\\
		(1-c)w_2\bar{L}_2(1-\beta)\\
		(1-c)w_2(x_2-\bar{U}_2\beta)
	\end{cases}\\
	\le (1-c)w_wx^0_2\le\\
	\begin{cases}
		w^Tx + cb + b(1-c)\beta - (1-c)w_1\bar{L}_1(1-\beta) - (1-c)\sum_{i>2}w_ix^0_i - y\\
		cw_1x_1 + \sum_{i>1}w_ix_i+cb+b(1-c)\beta + (1-c)\bar{U}_1\beta - (1-c)\sum_{i>2}w_ix^0_i-y\\
		(1-c)w_2\bar{U}_2(1-\beta)\\
		(1-c)w_2(x_2-\bar{L}_2\beta)
	\end{cases}
\end{split}
\end{align}
So we end up with the constraints
\begin{subequations}
\begin{align}
	y\ge cb + b(1-c)\beta + \sum_{i\in\mathcal{I}}w_ix_i - (1-c)(1-\beta)\sum_{i\in\mathcal{I}}w_i\bar{U}_i + c\sum_{i\notin\mathcal{I}}w_ix_i + (1-c)\beta\sum_{i\notin\mathcal{I}}w_i\bar{L}_i \label{eq:leaky_relu_nonextended_ideal_ge}\\
	y \le cb + b(1-c)\beta + \sum_{i\in\mathcal{I}}w_ix_i -(1-c)(1-\beta)\sum_{i\in\mathcal{I}}w_i\bar{L}_i + c\sum_{i\notin\mathcal{I}}w_ix_i + (1-c)\beta\sum_{i\notin\mathcal{I}}w_i\bar{U}_i \label{eq:leaky_relu_nonextended_ideal_le}
\end{align}
\end{subequations}
where $\mathcal{I}$ is a subset of $\{1, 2, ..., n\}$. Moreover, \eqref{eq:leaky_relu_nonextended_ideal_ge} is redundant.

In the end, we obtain the following non-extended ideal formulation
\begin{subequations}
\begin{align}
	y \ge c(w^Tx+b)\label{eq:leaky_relu_nonextended_ideal1}\\
	y \ge w^Tx+b\label{eq:leaky_relu_nonextended_ideal2}\\
	\begin{split}
		y \le cb + b(1-c)\beta + \sum_{i\in\mathcal{I}}\left(w_ix_i -(1-c)(1-\beta)w_i\bar{L}_i\right) +\\ \sum_{i\notin\mathcal{I}}\left(cw_ix_i + (1-c)\beta w_i\bar{U}_i\right)\label{eq:leaky_relu_nonextended_ideal3}
	\end{split}\\
	L\le x \le U\label{eq:leaky_relu_nonextended_ideal4}
\end{align}
\end{subequations}
Notice that there are exponential number of subsets $\mathcal{I}$. To find $\mathcal{I}$, we consider to take a point $\hat{x}, \hat{y}, \hat{\beta}$ satisfying the constraints in the big-M formulation, with $\hat{\beta}$ being the continuous relaxation $0 \le \hat{\beta} \le 1$. Then we take $\mathcal{I}$ as
\begin{align}
	\mathcal{I} = \{i|w_i\hat{x}_i \le (1-\hat{\beta})w_i\bar{L}_i + \hat{\beta}w_i\bar{U}_i\}\label{eq:find_index_set}
\end{align}

We take an iterative procedure to strengthen the big-M formulation. Say in the current iterations, the strengthened big-M formulation already contains the constraint with $\mathcal{I}_1,\hdots, \mathcal{I}_k$, then for an $\hat{x}$ at the boundary of the box $L \le x \le U$, we first compute its range of $\hat{\beta}$ under the existing constraint
\begin{align}
	\max(c(w^T\hat{x}+b), w^T\hat{x}+b) \le cb + b(1-c)\hat{\beta} + \sum_{i\in\mathcal{I}_k}(w_ix_i - (1-c)(1-\hat{\beta})w_i\bar{L}_i) + \sum_{i\notin\mathcal{I}_k}(cw_ix_i+(1-c)\hat{\beta}w_i\bar{U}_i)
\end{align}
Then for each extreme $\hat{x},\hat{\beta}$, we compute the corresponding index set $\mathcal{I}$ from \eqref{eq:find_index_set}. We then decide whether to enforce this cutting plane, by whether this new cutting plane shrinks the righ-hand side of \eqref{eq:leaky_relu_nonextended_ideal3} at $\hat{x}, \hat{\beta}$.

\section{Adding bounds on ReLU input}
If the lower and upper bound of the ReLU input $m^+/m^-$ is not computed as $\max/\min_{L\le x\le U} w^Tx+b$ (for example, we compute the bounds by solving an LP/MIP), then the leaky ReLU constraint $y = \max(cw^Tx+b, w^Tx+b)$ has the following ideal extended formulation
\begin{subequations}
	\begin{align}
	x = x^0 + x^1\\
	y = y^0 + y^1\\
	y^0 = c(w^Tx^0 + b(1-\beta))\le 0\\
	y^1 = w^Tx^1+b\beta \ge 0\\
	(m^--b)(1-\beta) \le w^Tx^0 \le(m^+-b)(1-\beta)\\
	(m^--b)\beta\le w^Tx^1\le(m^+-b)\beta\\
	L(1-\beta)\le x^0\le U(1-\beta)\\
	L\beta\le x^1\le U\beta
\end{align}
\end{subequations}
Hence we have the condition
\begin{align}
	\begin{split}
	\begin{cases}
		(1-c)(m^--b)(1-\beta) - (1-c)\sum_{i>1}w_ix^0_i\\
		(1-c)w^Tx - (1-c)(m^+-b)\beta - (1-c)\sum_{i>1}w_ix^0_i\\
		(1-c)w_1\bar{L}_1(1-\beta)\\
		(1-c)w_1(x_1-\bar{U}_1\beta)
	\end{cases}\\
	\le (1-c)w_1x^0_1\le\\
	\begin{cases}
		(1-c)(m^+-b)(1-\beta) - (1-c)\sum_{i>1}w_ix^0_i\\
		(1-c)w^Tx - (1-c)(m^--b)\beta - (1-c)\sum_{i>1}w_ix^0_i\\
		(1-c)w_1\bar{U}_1(1-\beta)\\
		(1-c)w_1(x_1-\bar{L}_1\beta)
	\end{cases}
\end{split}
\end{align}
Using the relationship
\begin{align}
	(1-c)w_1x^0_1 = w^Tx - (1-c)\sum_{i>1}w_ix^0_i + cb  + b(1-c)\beta - y\label{eq:1_c_w1_x01}
\end{align}
we get
\begin{align}
	\begin{split}
	\begin{cases}
		w^Tx + cb + b(1-c)\beta - (1-c)w_1\bar{U}_1(1-\beta)-(1-c)\sum_{i>1}w_ix^0_i\\
		cw_1x_1 + \sum_{i>1}w_ix_i + cb + b(1-c)\beta + (1-c)w_1\bar{L}_1\beta - (1-c)\sum_{i>1}w_ix^0_i\\
		w^Tx+b - (1-\beta)(1-c)m^+\\
		c(w^Tx+b) + m^-\beta(1-c)
	\end{cases}\\
	\le y \le\\
	\begin{cases}
		w^Tx + cb + b(1-c)\beta - (1-c)w_1\bar{L}_1(1-\beta)-(1-c)\sum_{i>1}w_ix^0_i\\
		cw_1x_1 + \sum_{i>1}w_ix_i + cb + b(1-c)\beta + (1-c)w_1\bar{U}_1\beta - (1-c)\sum_{i>1}w_ix^0_i\\
		w^Tx+b - (1-\beta)(1-c)m^-\\
		c(w^Tx+b) + m^+\beta(1-c)
	\end{cases}
\end{split}
\end{align}


\section{Appendix}
In this appendix, we try to derive the ideal non-extended formulation for ReLU unit (not the leaky ReLU) unit 
\begin{subequations}
\begin{align}
	y = \max(0, w^Tx + b)\\
	L \le x \le U
\end{align}
\end{subequations}
\subsection{big-M formulation}
We denote $m^+ = \sum_i \max(w_iL_i, w_iU_i), m^- = \sum_i \min(w_iL_i, w_iU_i)$, and get the following big-M formulation
\begin{subequations}
\begin{align}
	y \ge 0\\
	y \ge w^Tx + b\\
	y \le m^+\beta\\
	y \le w^Tx+b - m^-(1-\beta)\\
	L \le x \le U
\end{align}
\end{subequations}
\subsection{Ideal extended formulation}
Using the multiple-choice trick, we get the following ideal formulation with slack variables
\begin{subequations}
\begin{align}
	x = x^0 + x^1\\
	y = y^0 + y^1\\
	y^0 = 0 \ge w^Tx^0 + b(1-\beta)\\
	y^1 = w^Tx^1 + b\beta\ge 0\\
	L(1-\beta)\le x^0\le U(1-\beta)\\
	L\beta\le x^1\le U\beta
\end{align}
\end{subequations}

\subsection{Ideal non-extended formulation}
To derive the ideal non-extended formulation, we will need to remove the slack variable $x^0, x^1, y^0, y^1$ from the ideal extended formulation. To do so, we first remove $x^1, y^0, y^1$ as
\begin{align}
	x^1 = x - x^0\\
	y^0 = 0\\
	y^1 = w^T(x-x^0) + b\beta = y
\end{align}
Hence we have
\begin{align}
	w^Tx^0 = w^Tx + b\beta-y\\
	\Leftrightarrow w_1x^0_1 = w^Tx + b\beta -\sum_{i>1}w_ix^0_i - y\label{eq:w1x01}
\end{align}
Without loss of generality, we assume $w_i > 0\; \forall i$, hence
\begin{subequations}
\begin{align}
	w_1L_1(1-\beta)\le w_1x^0_1\le w_1U_1(1-\beta)\\
	w_1x_1 - w_1U_1\beta \le w_1x^0_1 \le w_1x_1 - w_1L_1\beta
\end{align}
\label{eq:w1x01_bounds}
\end{subequations}
Substituting $w_1x^0_1$ in the inequalities \eqref{eq:w1x01_bounds} with the right hand side of \eqref{eq:w1x01}, we get the following inequalities
\begin{subequations}
	\begin{align}
		y \ge w^Tx + b\beta - \sum_{i>1}w_ix^0_i - w_1U_1(1-\beta)\\
		y \ge \sum_{i>1}w_ix_i + b\beta - \sum_{i>1}w_ix^0_i + w_1L_1\beta\\
		y \le w^Tx + b\beta - \sum_{i>1}w_ix^0_i - w_1L_1(1-\beta)\\
		y \le \sum_{i>1}w_ix_i + b\beta - \sum_{i>1}w_ix^0_i  + w_1U_1\beta
	\end{align}
\end{subequations}
these inequalities remove the slack variable $x^0_1$.

Now we proceed to remove the slack variable $x^0_2$, we have
\begin{align}
	\begin{split}
	\max\begin{cases}
		w^Tx + b\beta -\sum_{i>2}w_ix^0_i - y - w_1U_1(1-\beta)\\
		\sum_{i>1}w_ix_i + b\beta - \sum_{i>2}w_ix^0_i -y + w_1L_1\beta\\
		w_2L_2(1-\beta)\\
		w_2x_2 - w_2U_2\beta
	\end{cases}\\
	\le w_2x^0_2\le\\
	\min\begin{cases}
		w^Tx + b\beta - \sum_{i>2}w_ix^0_i-y - w_1L_1(1-\beta)\\
		\sum_{i>1}w_ix_i + b\beta - \sum_{i>2}w_ix^0_i-y+w_1U_1\beta\\
		w_2U_2(1-\beta)\\
		w_2x_2-w_2L_2\beta
	\end{cases}
\end{split}
\end{align}
we get the inequalities
\begin{subequations}
\begin{align}
	y \ge w^Tx + b\beta - \sum_{i>2}w_ix^0_i - \sum_{i \le 2}w_iU_i(1-\beta)\\
	y \ge \sum_{i\neq 2}w_ix_i + b\beta - \sum_{i>2}w_ix^0_i - w_1U_1(1-\beta) + w_2L_2\beta\\
	y \ge \sum_{i>1}w_ix_i + b\beta - \sum_{i>2}w_ix^0_i + w_1L_1\beta - w_2U_2(1-\beta)\\
	y \ge \sum_{i>2}w_ix_i + b\beta - \sum_{i>2}w_ix^0_i  + \sum_{i\le 2}w_iL_i\beta\\
	y \le w^Tx + b + b\beta - \sum_{i>2}w_ix^0_i - \sum_{i\le 2}w_iL_i(1-\beta)\\
	y \le \sum_{i>1}w_ix_i + b\beta - \sum_{i>2}w_ix^0_i  + w_1U_1\beta - w_2L_2(1-\beta)\\
	y \le \sum_{i\neq 2}w_ix_i + b\beta - \sum_{i>2}w_ix^0_i - w_1L_1(1-\beta) + w_2U_2\beta\\
	y \le \sum_{i > 2}w_ix_i + b\beta - \sum_{i>2}w_ix^0_i + \sum_{i\le 2}w_iU_i\beta
\end{align}
\end{subequations}
If we follow this procedure to remove the remaining slack variables, we end up with the inequalities
\begin{subequations}
	\begin{align}
		y \ge \sum_{i\in\mathcal{I}}w_ix_i + b\beta - \sum_{i\in\mathcal{I}}w_iU_i(1-\beta) + \sum_{i\notin\mathcal{I}}w_iL_i\beta\\
		y \le \sum_{i\in\mathcal{I}}w_ix_i + b\beta - \sum_{i\in\mathcal{I}}w_iL_i(1-\beta) + \sum_{i\notin\mathcal{I}}w_iU_i\beta
	\end{align}
\end{subequations}
where $I$ is a subset of the set $\{1, 2, \hdots, n\}$.



\begin{thebibliography}{9}
	\bibitem{Anderson2020}
	Ross Anderson, Joey Huchette, Christian Tjandraatmadja and Juan Pablo Vielma \textit{Strong mixed-integer programming formulations for trained neural networks}, Mathematical Programming, 2020
\end{thebibliography}
\end{document}
