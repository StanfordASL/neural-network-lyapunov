\documentclass{article}
\usepackage{amsmath, amsfonts}
\newtheorem{theorem}{Theorem}
\newtheorem{lemma}{Lemma}
\begin{document}
\section{Spring Loaded Inverted Pendulum Dynamics}
For a \textit{spring loaded inverted pendulum} (SLIP) model, it has two modes, the flight phase and the stance phase. Its equation of motion is
\begin{itemize}
	\item Flight phase, state being $\mathbf{x}_{fl} = [x, z, \dot{x}, \dot{z}]^T$.
		\begin{align}
			\frac{d}{dt}
			\begin{bmatrix} x\\z\\\dot{x}\\\dot{z} \end{bmatrix} = f_{fl}(\mathbf{x}_{fl})
			=\begin{bmatrix}\dot{x}\\\dot{z}\\0\\-g\end{bmatrix}
		\end{align}
	\item Stance phase, state being $\mathbf{x}_{st}=[r, \theta, \dot{r}, \dot{\theta}, x_{foot}]^T$.
		\begin{align}
			\frac{d}{dt}\begin{bmatrix}r\\\theta\\\dot{r}\\\dot{\theta}\\\dot{x}_{foot}\end{bmatrix} = f_{st}(\mathbf{x}_{st})
			=\begin{bmatrix}\dot{r}\\\dot{\theta}\\k(l_0-r)/m - g\cos\theta+r\dot{\theta}^2\\g/r\sin\theta - 2\dot{r}\dot{\theta}/r\\0\end{bmatrix}
		\end{align}
	\item touch down guard
		\begin{align}
			g_{TD}(\mathbf{x}, \mathbf{u}) = y -l_0\cos\theta \ge 0
		\end{align}
	\item lift off guard
		\begin{align}
			g_{LO}(\mathbf{x}) = l_0 - r \ge 0
		\end{align}
	\item touch down transition maps a pre impact state $\mathbf{x}_{TD}^-$ to a post-impact state $\mathbf{x}_{TD}^+$
		\begin{align}
			\mathbf{x}_{TD}^+ = \Delta_{TD}(\mathbf{x}_{TD}^-, \mathbf{u}) = \begin{bmatrix} l_0\\\theta\\ -\dot{x}\sin\theta+\dot{y}\cos\theta\\(-\dot{x}\cos\theta-\dot{y}\sin\theta)/l_0\\x+l_0\sin\theta\end{bmatrix}
		\end{align}
	\item lift off transition maps a pre-lift-off state $\mathbf{x}_{LO}^-$ to a post-lift-off state $\mathbf{x}_{LO}^+$
		\begin{align}
			\mathbf{x}_{LO}^+ = \Delta_{LO}(\mathbf{x}_{LO}^-) = \begin{bmatrix}x_{foot} - l_0\sin\theta\\ l_0\cos\theta\\-l_0\cos\theta\dot{\theta}-\dot{r}\sin\theta \\ -l_0\sin\theta\dot{\theta}+\dot{r}\cos\theta\end{bmatrix}
		\end{align}
\end{itemize}

\section{Return map}
We will build the apex-to-apex Poincar\'e return map. We denote $\mathbf{x}_{apex}[n]$ as the n'th state at the apex of the flight phase, and the apex map as $\mathbf{x}_{apex}[n+1] = \phi(\mathbf{x}_{apex}[n], \mathbf{u}[n])$. This return map is nonlinear, and we will linearize this return map around a nominal apex state $\mathbf{x}^*_{apex}$ and leg angle $\mathbf{u}^*$ to get a linear dynamic model. We will then build the linearized model at many linearization points, and approximate the nonlinear return map as hybrid linear systems.

In order to linearize the return map, we will use the following lemma
\begin{lemma}
	\label{lemma:gradient_x0}
	For a dynamical system $\dot{\mathbf{x}} = f(\mathbf{x})$ with initial state $\mathbf{x}_0$, at a fixed time $T$, the gradient of $\partial{\mathbf{x}(t)}/\partial \mathbf{x}_0$ can be computed through integrating the following ODE on the matrix $\partial\mathbf{x}/\partial\mathbf{x}_0$
	\begin{align}
		\frac{d}{dt}\frac{\partial\mathbf{x}}{\partial \mathbf{x}_0} = \frac{\partial f}{\partial\mathbf{x}}\frac{\partial \mathbf{x}}{\partial \mathbf{x}_0}
	\end{align}
	with the initial condition $\partial \mathbf{x}(0)/\partial\mathbf{x}_0 = I$.
\end{lemma}
\subsection{Linearization of return map}
In order to compute $\frac{\partial\phi}{\partial{\mathbf{x}}_{apex}}$ and $\frac{\partial\phi}{\partial \mathbf{u}}$, we would first compute the gradient of the pre-touch-down state. The pre-touchdown state $\mathbf{x}_{TD}^-$ is computed as
\begin{align}
	\mathbf{x}_{TD}^- = \int_0^{t_{TD}} f_{fl}(\mathbf{x}_{fl}) d\tau
\end{align}
Using Lemma \ref{lemma:gradient_x0}, we know
\begin{align}
	\frac{\mathbf{x}_{TD}^-}{\partial \mathbf{x}_{apex}} = \underbrace{I + \int_0^{t_{TD}} \frac{\partial f_{fl}}{\partial \mathbf{x}}\frac{\partial \mathbf{x}}{\partial \mathbf{x}_{apex}} d\tau}_{\left.\frac{\partial \mathbf{x}(t)}{\partial x_{apex}}\right|_{t=t_{TD}}} + f_{fl}(\mathbf{x}_{TD}^-)\frac{\partial t_{TD}}{\partial \mathbf{x}_{apex}}
\end{align}
Note that we have the term $f_{fl}(\mathbf{x}_{TD}^-)\frac{\partial t_{TD}}{\partial \mathbf{x}_{apex}}$ because the touch down time $t_{TD}$ is also a function of the initial apex state (and the control).

To compute the term $\frac{\partial t_{TD}}{\partial \mathbf{x}_{apex}}$, we notice that
\begin{align}
	\dot{g}_{TD} = \frac{\partial{g}_{TD}}{\partial \mathbf{x}}f_{fl}(\mathbf{x}_{TD}^-)
\end{align}
and 
\begin{align}
	\frac{\partial g_{TD}}{\partial \mathbf{x}_{apex}} = \frac{\partial g_{TD}}{\partial \mathbf{x}}\left.\frac{\partial \mathbf{x}(t)}{\partial \mathbf{x}_{apex}}\right|_{t = t_{TD}}
\end{align}
With the above two equations, we can compute $\frac{\partial t_{TD}}{\partial \mathbf{x}_{apex}}$ as
\begin{align}
	\frac{\partial t_{TD}}{\partial \mathbf{x}_{apex}} =& \frac{\partial t_{TD}}{\partial g_{TD}}\frac{\partial g_{TD}}{\partial \mathbf{x}_{apex}}\\
	=&\left(\frac{\partial{g}_{TD}}{\partial \mathbf{x}}f_{fl}(\mathbf{x}_{TD}^-)\right)^{-1}\frac{\partial g_{TD}}{\partial \mathbf{x}}\left.\frac{\partial \mathbf{x}(t)}{\partial \mathbf{x}_{apex}}\right|_{t = t_{TD}}
\end{align}

Similarly, we compute the gradient w.r.t $\mathbf{u}$ as
\begin{align}
	\frac{\partial \mathbf{x}_{TD}^-}{\partial \mathbf{u}} =& f_{fl}(\mathbf{x}_{TD}^-)\frac{\partial t_{TD}}{\partial \mathbf{u}}\\
	=&f_{fl}(\mathbf{x}_{TD}^-)\frac{\partial t_{TD}}{\partial g_{TD}}\frac{\partial g_{TD}}{\partial \mathbf{u}}\\
	=&f_{fl}(\mathbf{x}_{TD}^-)
\end{align}

We can then compute the gradient of the post-touch-down state $\mathbf{x}_{TD}^+$ w.r.t the initial apex state using chain rule as
\begin{align}
	\frac{\partial \mathbf{x}_{TD}^+}{\partial \mathbf{x}_{apex}} = \frac{\partial \Delta_{TD}}{\partial x}\frac{\partial \mathbf{x}_{TD}^-}{\partial \mathbf{x}_{apex}}\\
	\frac{\partial \mathbf{x}_{TD}^+}{\partial \mathbf{u}} = \frac{\partial \Delta_{TD}}{\partial x}\frac{\partial \mathbf{x}_{TD}^-}{\partial \mathbf{u}} + \frac{\partial \Delta_{TD}}{\partial \mathbf{u}}
\end{align}

\end{document}
